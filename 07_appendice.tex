% !TEX TS-program = lualatex
% !TEX encoding = UTF-8

\documentclass[Session2025.tex]{subfiles}

\ifcsname preamble@file\endcsname
  \setcounter{page}{\getpagerefnumber{M-07_appendice}}
\fi

\begin{document}

\addcontentsline{toc}{chapter}{Appendice}

\bigtitle{Te Deum}{Appendice}{Te Deum}
\smalltitle{ton monastique}
\gscore{hy_te_deum}

\newpage

\bigtitle{Ave Maris Stella}{Appendice}{Ave Maris Stella}
\smalltitle{ton solennel}
\gscore{hy_ave_maris_stella}

\pagebreak

\null\vfill

\bigtitle{Tons des psaumes}{Appendice}{Tons des psaumes}
\smalltitle{selon l'office monastique rénové}

\smalltitle{Ton 1}
\incipit{ton}{1}

\smalltitle{Ton 2}
\incipit{ton}{2}

\newpage

\smalltitle{Ton 2*}
\incipit{ton}{2s}

\smalltitle{Ton 3}
\incipit{ton}{3}

\smalltitle{Ton 4}
\incipit{ton}{4}

\newpage

\smalltitle{Ton 4*}
\incipit{ton}{4s}

\smalltitle{Ton 5}
\incipit{ton}{5}

\smalltitle{Ton 6}
\incipit{ton}{6}

\newpage

\smalltitle{Ton 7}
\incipit{ton}{7}

\smalltitle{Ton 8}
\incipit{ton}{8}

\smalltitle{Ton C, ou «in directum»}
\incipit{ton}{C}

\newpage 

\smalltitle{Ton D}
\incipit{ton}{D}

\smalltitle{Ton E, ou «irrégulier»}
\incipit{ton}{E}

\smalltitle{Ton «pérégrin»}
\incipit{ton}{P}

\end{document}