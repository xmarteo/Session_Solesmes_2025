% !TEX TS-program = lualatex
% !TEX encoding = UTF-8

\documentclass[Session2025.tex]{subfiles}

\ifcsname preamble@file\endcsname
  \setcounter{page}{\getpagerefnumber{M-03_mardi}}
\fi

\begin{document}
\addcontentsline{toc}{chapter}{Mardi 8 juillet, de la férie, messe votive des S. Anges}
\bigtitle{Messe votive des saints Anges}{mardi 8 juillet}{Messe}

\gscore{in_adorate_deum}
\translation{\aa TODO.\\
\vv TODO.}

\smalltitle{Psalmodie de Tierce}

\gscore{an_clamavi}
\translation{\aa J'ai crié, et il m'a exaucé.}
\smalltitle{Psaume 119}
\psalm{119}{E}
\smalltitle{Psaume 120}
\psalm{120}{E}
\needspace{3cm}
\smalltitle{Psaume 121}
\psalm{121}{E}

\gscore{ky_k15}
\gscore{gr_laudate_dominum}
\translation{\rr TODO.
\vv TODO.}
\gscore{al_angelus_domini_descendit}
\translation{\vv TODO.}
\gscore{of_stetit_angelus}
\translation{TODO.}

\gscore{ky_s15}
\gscore{ky_a15}

\gscore{co_benedicite_omnes_angeli}
\translation{\aa TODO.\\
\vv TODO.\\
\vv TODO.}

\bigtitle{Mardi II, à Sexte}{mardi 8 juillet}{Sexte}

\smallscore{or_dia_ferialis}

\gscore{hy_rector_potens_ferialis}
\translation{\colored{M}aître puissant, Dieu Vérité, tu règles la marche du temps, tu formes l’aube en sa clarté, au midi tu donnes ses flammes.\\
\colored{E}teins le feu des dissensions, calme la fièvre du péché, apporte à nos corps la santé, à nos cœurs, la paix véritable.\\
\colored{E}xauce-nous, Père très bon, et toi, le Fils égal au Père, avec l’Esprit Consolateur, régnant pour les siècles des siècles.}

\gscore{an_qui_habitas}
\translation{\aa Toi qui habites dans les cieux, aie pitié de nous.}
\smalltitle{Psaume 122}
\psalm{122}{8}
\smalltitle{Psaume 123}
\psalm{123}{8}
\smalltitle{Psaume 124}
\psalm{124}{8}

\capitulum{1 Cor 12, 12-13}
{Sicut corpus unum est et membra habet multa, † ómnia autem membra
córporis, cum sint multa, unum corpus sunt, * ita et Christus ; / étenim in
uno Spíritu omnes nos in unum corpus baptizáti sumus, † sive Iudǽi sive
Græci sive servi sive líberi, * et omnes unum Spíritum potáti sumus.}
{Prenons une comparaison : notre corps forme un tout, il a pourtant
plusieurs membres ; et tous les membres, malgré leur nombre, ne forment
qu’un seul corps. Il en est ainsi pour le Christ. Tous, Juifs ou païens,
esclaves ou hommes libres, nous avons été baptisés dans l’unique Esprit
pour former un seul corps. Tous nous avons été désaltérés par l’unique
Esprit.}

\versiculus{Pater sancte, serva nos in nómine tuo}{Ut simus consummáti in unum}{Père saint, garde-nous fidèles à ton nom}{Que notre unité soit parfaite}

\oratio{
Deus, qui Petro salvíficum tuum super gentes consílium
revelásti, éffice benígnus, ut ópera nostra tibi
grata reddántur, ac tuo dilectiónis salutísque propósito
te donánte desérviant. Per Christum.
}{
Dieu qui as révélé à l’Apôtre
Pierre ta volonté de sauver
tous les hommes, accorde-nous
de déployer toutes nos énergies
au service de ce dessein de ton
amour. Par le Christ.
}

\gscore{or_benedicamus_hm_ferialis}

\bigtitle{Mardi II, à None}{mardi 8 juillet}{None}

\smallscore{or_dia_ferialis}

\gscore{hy_rerum_deus_ferialis}
\translation{\colored{D}ieu fort, soutien de l’univers, qui es en toi sans changement, tu fixes dans leur succession les temps que le soleil mesure.\\
\colored{D}onne-nous un soir lumineux où la vie ne décline pas, où, pour fruit de la Sainte Mort, resplendira toujours la gloire.\\
\colored{E}xauce-nous, Père très bon, et toi, le Fils égal au Père, avec l’Esprit Consolateur, régnant pour les siècles des siècles.}

\gscore{an_beati_omnes}
\translation{\aa Heureux tous ceux qui craignent le Seigneur.}
\smalltitle{Psaume 125}
\psalm{125}{2}
\smalltitle{Psaume 126}
\psalm{126}{2}
\smalltitle{Psaume 127}
\psalm{127}{2}

\capitulum{1 Cor 12, 24b.25-26}
{Deus temperávit corpus, ut non sit schisma in córpore, sed idípsum
pro ínvicem sollícita sint membra. / Et sive pátitur unum membrum,
compatiúntur ómnia membra ; * sive glorificátur unum membrum,
congáudent ómnia membra.}
{Dieu a organisé le corps de telle façon qu’il n’y ait pas de division dans
le corps, mais que les différents membres aient tous le souci les uns des
autres. Si un membre souffre, tous les membres partagent sa souffrance ;
si un membre est à l’honneur, tous partagent sa joie.}

\versiculus{Dómine Deus noster, cóngrega nos de natiónibus}{Ut confiteámur nómini sancto tuo}{Rassemble-nous, Seigneur, du milieu des nations}{Que nous rendions grâce à ton saint nom}

\oratio{
Deus, qui Cornélio centurióni ángelum tuum misísti,
ut viam ei salútis osténderet, da nobis, quǽsumus,
in salvatiónem ómnium libéntius operári, ut una cum
eis, in Ecclésia tua, ad te perveníre possímus. Per Christum.
}{
Dieu, qui as envoyé ton ange
au centurion Corneille
pour lui montrer le bon chemin,
donne-nous de travailler au
salut du monde : qu’avec l’humanité
tout entière, en communion
à ton Église, nous parvenions jusqu’à
toi. Par le Christ.
}

\gscore{or_benedicamus_hm_ferialis}

\bigtitle{Mardi II, aux Vêpres}{mardi 8 juillet}{Vêpres}

\smallscore{or_dia_ferialis}

\smalltitle{Psaume 129}
\gscore{an_de_profundis}
\translation{\aa Du fond des abîmes je crie vers toi, Seigneur.}
\psalm{129}{8}

\smalltitle{Psaume 130}
\gscore{an_speret_israel}
\translation{\aa Espère, Israël, dans le Seigneur.}
\psalm{130}{E}

\smalltitle{Psaume 131}
\gscore{an_et_omnis_mansuetudinis}
\translation{\aa Et toutes ses mansuétudes.}
\psalm{131}{E}

\smalltitle{Psaume 132}
\gscore{an_habitare_fratres}
\translation{\aa Pour des frères, habiter ensemble.}
\psalm{132}{1}

\capitulum{Rom 3, 23-25a}
{Omnes peccavérunt et egent glória Dei, † iustificáti gratis per grátiam
ipsíus per redemptiónem quæ est in Christo Iesu ; * quem propósuit Deus
propitiatórium per fidem in sánguine ipsíus ad ostensiónem iustítiæ suæ.}
{Tous les hommes sont pécheurs, ils sont tous privés de la gloire de Dieu,
lui qui leur donne d’être des justes par sa seule grâce, en vertu de la
rédemption accomplie dans le Christ Jésus. Car Dieu a exposé le Christ
sur la croix afin que, par l’offrande de son sang, il soit le pardon pour
ceux qui croient en lui.}

\gscore{rb_adimplebis}
\translation{\rr Il y a plénitude de joie * devant ta face, Seigneur.\\
\vv Des délices éternels dans Ta droite jusqu'à la fin.}

\gscore{hy_sator_princepsque_temporum}
\translation{\colored{C}réateur et Maître des temps,
tu fixes l’ordre qui distingue,
pour le travail, le jour qui luit,
et la nuit, pour notre sommeil.\\
\colored{D}aigne conduire l’âme pure :
que le silence de la nuit
n’expose nos coeurs au Jaloux
qui veut les blesser de ses traits.\\
\colored{L}oin de la fièvre des passions
puissent-ils ne jamais brûler
des feux qui, s’attachant aux sens,
épuisent la vigueur de l’âme.\\
\colored{E}xauce-nous, Père très bon,
et toi, le Fils égal au Père,
avec l’Esprit Consolateur,
régnant pour les siècles des siècles.}

\versiculus{Vespertína orátio ascéndat ad te, Dómine}{Et descéndat super nos misericórdia tua}{Que la prière du soir monte vers toi, Seigneur}{Et que descende sur nous ta miséricorde}

\smalltitle{Magnificat}
\gscore{an_respexit_dominus}
\translation{\aa Le Seigneur a vu ma petitesse, et le puissant a fait pour moi des merveilles.}
\incipit{magn}{8g}
\psalm{magn}{8}

\smalltitle{Intercession}
\smallscore{or_kyrie_ferialis}
\smallscore{or_pater_ferialis}

\oratio{
Dómine Deus, cuius est dies et cuius est nox,~\pscross{} concéde
solem iustítiæ in nostris córdibus permanére,~\psstar{} ut ad
lucem, quam inhábitas, perveníre possímus. Per Dóminum.
}{
Seigneur notre Dieu, à qui appartiennent
et le jour et la
nuit, fais demeurer en nos coeurs
le Soleil de justice, afin que nous
puissions parvenir à la lumière
que tu habites. Par Jésus Christ.
}

\blessing

\gscore{or_benedicamus_feriis}

\end{document}
