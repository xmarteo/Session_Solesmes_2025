% !TEX TS-program = lualatex
% !TEX encoding = UTF-8

\documentclass[Session2025.tex]{subfiles}

\ifcsname preamble@file\endcsname
  \setcounter{page}{\getpagerefnumber{M-03_mardi}}
\fi

\begin{document}
\addcontentsline{toc}{chapter}{Mardi 8 juillet, de la férie, messe votive des S. Anges}
\bigtitle{Messe votive des saints Anges}{mardi 8 juillet}{Messe}

\gscore{in_cognovi_domine}
\translation{\aa J'ai reconnu, Seigneur, que tes jugements sont équitables, et que tu m'as humilié selon ta justice.
Transperce ma chair par ta crainte; je redoute tes jugements.\\
\vv Heureux ceux qui sont immaculés dans la voie, qui marchent dans la loi du Seigneur.}

\smalltitle{Psalmodie de Tierce}

\gscore{an_clamavi}
\translation{\aa J'ai crié, et il m'a exaucé.}
\smalltitle{Psaume 119}
\psalm{119}{E}
\smalltitle{Psaume 120}
\psalm{120}{E}
\needspace{3cm}
\smalltitle{Psaume 121}
\psalm{121}{E}

\gscore{ky_k15}
\gscore{ky_g15}
\gscore{gr_diffusa_est_gratia}
\translation{\rr La grâce est répandue sur tes lèvres; c'est pourquoi Dieu t'a bénie à jamais.
\vv Pour la vérité, la douceur et la justice; et ta droite te conduira merveilleusement.}
\gscore{al_specie_tua}
\translation{\vv Avec ta gloire et ta majesté, avance, marche victorieusement, et règne.}
\gscore{of_diffusa_est_gratia}
\translation{\rr La grâce est répandue sur tes lèvres; c'est pourquoi Dieu t'a bénie à jamais, dans les siècles.}
\newpage
\gscore{ky_s15}
\gscore{ky_a15}
\newpage
\gscore{co_dilexisti_iustitiam}
\translation{\aa Tu as aimé la justice, et haï l'iniquité; c'est pourquoi Dieu t'a ointe d'une huile d'allégresse d'une manière plus excellente que toutes tes compagnes.\\
\vv D'heureuses paroles jaillissent de mon coeur quand je dis mes poèmes pour le roi.\\
\vv Écoute, ma fille, regarde et tends l'oreille ; oublie ton peuple et la maison de ton père.}

\bigtitle{Mardi II, à Sexte}{mardi 8 juillet}{Sexte}

\smallscore{or_dia_ferialis}

\gscore{hy_rector_potens_ferialis}
\translation{\colored{M}aître puissant, Dieu Vérité, tu règles la marche du temps, tu formes l’aube en sa clarté, au midi tu donnes ses flammes.\\
\colored{E}teins le feu des dissensions, calme la fièvre du péché, apporte à nos corps la santé, à nos cœurs, la paix véritable.\\
\colored{E}xauce-nous, Père très bon, et toi, le Fils égal au Père, avec l’Esprit Consolateur, régnant pour les siècles des siècles.}

\gscore{an_qui_habitas}
\translation{\aa Toi qui habites dans les cieux, aie pitié de nous.}
\smalltitle{Psaume 122}
\psalm{122}{8}
\smalltitle{Psaume 123}
\psalm{123}{8}
\smalltitle{Psaume 124}
\psalm{124}{8}

\capitulum{1 Cor 12, 12-13}
{Sicut corpus unum est et membra habet multa, † ómnia autem membra
córporis, cum sint multa, unum corpus sunt, * ita et Christus ; / étenim in
uno Spíritu omnes nos in unum corpus baptizáti sumus, † sive Iudǽi sive
Græci sive servi sive líberi, * et omnes unum Spíritum potáti sumus.}
{Prenons une comparaison : notre corps forme un tout, il a pourtant
plusieurs membres ; et tous les membres, malgré leur nombre, ne forment
qu’un seul corps. Il en est ainsi pour le Christ. Tous, Juifs ou païens,
esclaves ou hommes libres, nous avons été baptisés dans l’unique Esprit
pour former un seul corps. Tous nous avons été désaltérés par l’unique
Esprit.}

\versiculus{Pater sancte, serva nos in nómine tuo}{Ut simus consummáti in unum}{Père saint, garde-nous fidèles à ton nom}{Que notre unité soit parfaite}

\rubric{Oraison TODO}

\gscore{or_benedicamus_hm_ferialis}

\bigtitle{Mardi II, à None}{mardi 8 juillet}{None}

\smallscore{or_dia_ferialis}

\gscore{hy_rerum_deus_ferialis}
\translation{\colored{D}ieu fort, soutien de l’univers, qui es en toi sans changement, tu fixes dans leur succession les temps que le soleil mesure.\\
\colored{D}onne-nous un soir lumineux où la vie ne décline pas, où, pour fruit de la Sainte Mort, resplendira toujours la gloire.\\
\colored{E}xauce-nous, Père très bon, et toi, le Fils égal au Père, avec l’Esprit Consolateur, régnant pour les siècles des siècles.}

\gscore{an_beati_omnes}
\translation{\aa Heureux tous ceux qui craignent le Seigneur.}
\smalltitle{Psaume 125}
\psalm{125}{2}
\smalltitle{Psaume 126}
\psalm{126}{2}
\smalltitle{Psaume 127}
\psalm{127}{2}

\capitulum{1 Cor 12, 24b.25-26}
{Deus temperávit corpus, ut non sit schisma in córpore, sed idípsum
pro ínvicem sollícita sint membra. / Et sive pátitur unum membrum,
compatiúntur ómnia membra ; * sive glorificátur unum membrum,
congáudent ómnia membra.}
{Dieu a organisé le corps de telle façon qu’il n’y ait pas de division dans
le corps, mais que les différents membres aient tous le souci les uns des
autres. Si un membre souffre, tous les membres partagent sa souffrance ;
si un membre est à l’honneur, tous partagent sa joie.}

\versiculus{Dómine Deus noster, cóngrega nos de natiónibus}{Ut confiteámur nómini sancto tuo}{Rassemble-nous, Seigneur, du milieu des nations}{Que nous rendions grâce à ton saint nom}

\rubric{Oraison TODO}

\gscore{or_benedicamus_hm_ferialis}

\bigtitle{Mardi II, aux Vêpres}{mardi 8 juillet}{Vêpres}

\smallscore{or_dia_ferialis}

\gscore{hy_sator_princepsque_temporum}
\translation{\colored{L}TODO.\\
\colored{B}lessée par l’amour divin, elle a foulé aux pieds les biens éphémères du siècle pour parcourir le chemin escarpé du ciel.\\
\colored{E}n domptant sa chair par le jeûne et en nourrissant son âme des délices de la prière, elle a acquis les joies célestes.\\
\colored{C}hrist Roi, force des forts, unique auteur de la grandeur, par l’intercession de cette sainte, écoute avec bonté nos supplications.\\
\colored{G}loire à toi, Jésus, qui nous donnes d’espérer les suffrages de la bienheureuse servante et les récompenses éternelles.}

\smalltitle{Psaume 109}
\gscore{an_accinxit_fortitudine}
\translation{\aa Elle s'est ceinte de force, et a rendu fort son bras: c'est pouquoi sa lampe ne s'éteindra pas, éternellement.}
\psalm{109}{8}

\smalltitle{Psaume 112}
\gscore{an_via_iustorum}
\translation{\aa Le chemin des justes a été rendu droit, et le sentier des saints a été préparé.}
\psalm{112}{8}

\smalltitle{Psaume 121}
\gscore{an_cognovit_eam}
\translation{\aa Le Seigneur l'a connue dans ses bénédictions, et elle a trouvé grâce auprès du Seigneur.}
\psalm{121}{8}

\smalltitle{Psaume 126}
\gscore{an_lauda_qui_posuit}
\translation{\aa Célèbre le Seigneur, Jérusalem, car il a mis la paix à tes frontières, et en toi il a béni tes fils.}
\psalm{126}{7}

\capitulum{Rom 3, 23-25a}
{Omnes peccavérunt et egent glória Dei, † iustificáti gratis per grátiam
ipsíus per redemptiónem quæ est in Christo Iesu ; * quem propósuit Deus
propitiatórium per fidem in sánguine ipsíus ad ostensiónem iustítiæ suæ.}
{Tous les hommes sont pécheurs, ils sont tous privés de la gloire de Dieu,
lui qui leur donne d’être des justes par sa seule grâce, en vertu de la
rédemption accomplie dans le Christ Jésus. Car Dieu a exposé le Christ
sur la croix afin que, par l’offrande de son sang, il soit le pardon pour
ceux qui croient en lui.}

\gscore{rb_TODO}
\translation{\rr Dieu l'a choisie, il l'a prédestinée.\\
\vv Il l'a fait habiter dans sa demeure.}

\newpage

\smalltitle{Magnificat}
\gscore{an_ego_dominus_sponsabo_te}
\translation{\aa Moi, le Seigneur, je te fiancerai à moi pour toujours,
je te fiancerai à moi dans la justice et le droit, dans la fidélité et la tendresse ;
je te fiancerai à moi dans la loyauté.}
\incipit{magn}{7dsol}
\psalm{magn}{7sol}

\smalltitle{Intercession}
\smallscore{or_kyrie_ferialis}
\smallscore{or_pater_ferialis}

\needspace{2cm}
\label{0723V}
\oratio{
TODO}{
Seigneur Dieu, tu as conduit sainte Brigitte par divers chemins de vie,
et tu lui as enseigné de façon admirable la sagesse de la croix par la
contemplation de la Passion de ton Fils ; accorde à chacun de nous, quel
que soit son état de vie, de savoir te chercher en toute chose.
}

\blessing

\gscore{or_benedicamus_feriis}

\end{document}
