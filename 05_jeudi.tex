% !TEX TS-program = lualatex
% !TEX encoding = UTF-8

\documentclass[Session2025.tex]{subfiles}

\ifcsname preamble@file\endcsname
  \setcounter{page}{\getpagerefnumber{M-05_jeudi}}
\fi

\begin{document}
\addcontentsline{toc}{chapter}{Jeudi 10 juillet, de la férie}
\bigtitle{Jeudi de la 14\textsuperscript{e} semaine \emph{per annum},\\à la Messe}{jeudi 10 juillet}{Messe}

\rubric{Introït \normaltext{Suscépimus}, p. \pageref{in_suscepimus}.}

\rubric{Psalmodie de Tierce, comme au mardi, p. \pageref{an_clamavi}.}

\rubric{Kyriale XVI, page \pageref{ky_k16}.}

\gscore{gr_esto_mihi}
\translation{\rr Sois pour moi un Dieu protecteur et un lieu de refuge, afin de me sauver.\\
\vv Dieu, j'espère en toi ; Seigneur, je ne serai pas confondu à jamais.}

\rubric{Alléluia, Offertoire et Communion comme au lundi, page \pageref{al_magnus_dominus}.}

\bigtitle{Jeudi II, à Sexte}{jeudi 10 juillet}{Sexte}

\rubric{Ouverture, Hymne et Psalmodie de Sexte, comme au mardi, p. \pageref{hy_rector_potens_ferialis}.}

\capitulum{Gal 5, 16-17}
{Spíritu ambuláte et concupiscéntiam carnis ne perfecéritis. / Caro enim
concupíscit advérsus Spíritum, Spíritus autem advérsus carnem ; † hæc
enim ínvicem adversántur, * ut non quæcúmque vultis illa faciátis.}
{Marchez sous la conduite de
l’Esprit Saint, et vous ne
risquerez pas de satisfaire les
convoitises de la chair. Car les
tendances de la chair s’opposent
à l’Esprit, et les tendances de
l’Esprit s’opposent à la chair. En
effet, il y a là un affrontement
qui vous empêche de faire tout
ce que vous voudriez.}

\versiculus{Bonus es tu, Dómine, et benefáciens}{Doce me iustificatiónes tuas}{Toi, tu es bon, Seigneur, tu fais du bien}{Apprends-moi tes commandements}

\oratio{
Omnípotens sempitérne Deus, apud quem nihil est
tenebrósum, nihil obscúrum,~\pscross{} lucis tuæ in nos emítte
splendórem,~\psstar{} ut mandatórum tuórum lege percépta, in
via tua dilatáto corde fidéliter ambulémus. Per Christum.
}{
Dieu éternel et tout-puissant,
en qui rien n’est sombre ni
obscur, communique ta lumière
à nos coeurs : en recevant ta loi et
tes préceptes, nous marcherons
sur ta route d’un coeur léger. Par
le Christ.
}

\gscore{or_benedicamus_hm_ferialis}

\bigtitle{Jeudi II, à None}{jeudi 10 juillet}{None}

\rubric{Ouverture, Hymne et Psalmodie de None, comme au mardi, p. \pageref{hy_rerum_deus_ferialis}.}

\capitulum{Gal 5, 22-23a.25}
{Fructus Spíritus est cáritas, gáudium, pax, longanímitas, * benígnitas,
bónitas, fides, mansuetúdo, continéntia. / Si vívimus Spíritu, Spíritu et
ambulémus.}
{Voici le fruit de l’Esprit :
amour, joie, paix, patience,
bonté, bienveillance, fidélité,
douceur et maîtrise de soi.
Puisque l’Esprit nous fait vivre,
marchons sous la conduite de
l’Esprit.}

\versiculus{Notam fac mihi, Dómine, viam in qua ámbulem}{Spíritus tuus bonus dedúcat me in terram rectam}{Montre-moi le chemin que je dois prendre}{Ton souffle est bienfaisant : qu’il me guide en un pays de plaines}

\oratio{
Da nobis orántibus, quǽsumus, Dómine,~\pscross{} ut patiéntiæ
Unigéniti tui sequámur exémpla,~\psstar{} et advérsa
patiéndi constántiam habeámus. Per Christum.
}{
Nous t’en prions, Seigneur,
que la passion de ton Fils
unique demeure devant nos
yeux et nous fortifie dans les
épreuves. Par le Christ.
}

\gscore{or_benedicamus_hm_ferialis}

\needspace{5cm}
\bigtitle{Saint Benoît, aux 1\textsuperscript{e} Vêpres}{jeudi 10 juillet}{Vêpres}

\smallscore{or_dia_festivus}

\smalltitle{Psaume 112}
\gscore{an_fuit_vir}
\translation{\aa Il y eut un homme d'une vie vénérable, Benoît (béni) de grâce comme de nom, qui, depuis le temps de son enfance, eut le coeur d'un vieillard et, dépassant son âge par sa manière de vivre, ne voua son esprit à aucune volupté.}
\psalm{112}{8}

\smalltitle{Psaume 145}
\gscore{an_beatus_vir}
\translation{\aa Le bienheureux Benoît désira davantage de souffrir les maux du monde que d'en recevoir les louanges, de s'épuiser en travaux pour Dieu que d'être flatté par les plaisirs de cette vie.}
\psalm{145}{3}

\smalltitle{Psaume 146}
\gscore{an_gloriosus_confessor}
\translation{\aa Benoît, glorieux confesseur du Seigneur, menant une vie angélique sur la terre, est devenu un miroir des bonnes oeuvres pour le monde ; c'est pourquoi désormais il se réjouit sans fin avec le Christ dans le ciel.}
\psalm{146}{8}

\smalltitle{Psaume 147}
\gscore{an_vir_domini}
\translation{\aa L'homme du Seigneur Benoît fut rempli de l'esprit de tous les justes : qu'il intercède pour tous les moines.}
\psalm{147}{1}

\capitulum{Rom 8, 28-30}
{Scimus quóniam diligéntibus Deum ómnia cooperántur in bonum,~\pscross{} his, qui secúndum propósitum vocáti sunt. / Nam, quos præscívit, et prædestinávit confórmes fíeri imáginis Fílii eius,~\pscross{} ut sit ipse primogénitus in multis frátribus;~\psstar{} quos autem prædestinávit, hos et vocávit; et quos vocávit, hos et iustificávit; quos autem iustificávit, illos et glorificávit.}
{Nous le savons, quand les
hommes aiment Dieu,
lui-même fait tout contribuer
à leur bien, puisqu’ils sont appelés
selon le dessein de son
amour. Ceux que, d’avance, il
connaissait, il les a aussi destinés
d’avance à être configurés
à l’image de son Fils, pour que
ce Fils soit le premier-né d’une
multitude de frères. Ceux qu’il
avait destinés d’avance, il les
a aussi appelés ; ceux qu’il a appelés,
il en a fait des justes ; et
ceux qu’il a rendus justes, il leur
a donné sa gloire.}

\gscore{resp_amavit_eum}
\translation{\rr Le Seigneur l’a aimé et l’a paré ; il l’a revêtu d’une robe de gloire : * Et il l’a couronné aux portes du paradis.
\vv Le Seigneur l’a revêtu de la cuirasse de la foi et il l’a paré.}

\gscore{hy_fratres_alacri_pectore}
\translation{\colored{F}rères, venez avec un cœur joyeux, partageons d'une même voix les joies de cette glorieuse fête.\\
\colored{E}n ce jour, Benoît, révélateur du chemin étroit, reçoit la récompense de ses travaux et se réjouit de l'honneur qui lui est dû.\\
\colored{I}l a brillé comme une nouvelle étoile, dissipant les nuages terrestres ; dès le seuil de l'âge, il a méprisé les plaisirs de la jeunesse.\\
\colored{P}uissant en miracles, inspiré par le souffle du Très-Haut, il a resplendi par des prodiges, annonçant les choses futures à son siècle.\\
\colored{P}armi ces œuvres, il a également brillé par ses enseignements lumineux, car il a tracé avec justesse la voie de la vie sacrée pour ceux qui le suivent.
\colored{G}loire à la Trinité, dont la grâce remplit ce saint de nombreuses fleurs de sainteté, embellissant ainsi les cieux.}

\versiculus{Sit memória eius in benedictióne}{Et pérmanens ad fílios viri sancti glória}{Que sa mémoire soit bénie}{Et que la gloire de ce père saint se continue dans ses fils}

\smalltitle{Magnificat}
\gscore{an_exsultet_omnium}
\translation{\aa Qu'exulte la foule des fidèles en raison de la gloire du bon Père Benoît. Que se réjouissent en particulier les troupes de moines qui célèbre sur terre la fête de celui de la compagnie duquel les saints se réjouissent au ciel.}
\incipit{magn}{1dsol}
\psalm{magn}{1sol}

\smalltitle{Intercession}
\smallscore{or_kyrie_festivus}
\smallscore{or_pater_festivus}

\label{0711_oratio}
\oratio{
Deus, qui beátum Patrem nostrum Benedíctum abbátem in schola divíni servítii præclárum constituísti magístrum, tríbue, quǽsumus, ut, amóri tuo nihil præponéntes, viam mandatórum tuórum dilatáto corde currámus. Per Dóminum.
}{
Dieu, qui as établi notre bienheureux Père Benoît abbé comme un maître éminent dans l'école du service divin, accorde-nous, nous T'en prions, de ne rien préférer à Ton amour, afin que nous courions avec un cœur dilaté sur la voie de Tes commandements.
}

\blessing

\gscore{or_benedicamus_sollemnitatibus}

\end{document}