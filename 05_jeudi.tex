% !TEX TS-program = lualatex
% !TEX encoding = UTF-8

\documentclass[Session2025.tex]{subfiles}

\ifcsname preamble@file\endcsname
  \setcounter{page}{\getpagerefnumber{M-05_jeudi}}
\fi

\begin{document}
\addcontentsline{toc}{chapter}{Jeudi 10 juillet, de la férie}
\bigtitle{Jeudi de la 14\textsuperscript{e} semaine \emph{per annum, à la Messe}{jeudi 10 juillet}{Messe}

\rubric{Introït \normaltext{Suscépimus}, p. \pageref{in_suscepimus}.}

\rubric{Psalmodie de Tierce, comme au mardi, p. \pageref{an_clamavi}.}

\rubric{Kyriale XVI, page \pageref{ky_k16}.}

\gscore{gr_esto_mihi}
\translation{\rr Sois pour moi un Dieu protecteur et un lieu de refuge, afin de me sauver.\\
\vv Dieu, j'espère en toi ; Seigneur, je ne serai pas confondu à jamais.}

\rubric{Alléluia, Offertoire et Communion comme au lundi, page \pageref{al_magnus_dominus}.}

\bigtitle{Jeudi II, à Sexte}{jeudi 10 juillet}{Sexte}

\rubric{Ouverture, Hymne et Psalmodie de Sexte, comme au mardi, p. \pageref{hy_rector_potens_ferialis}.}

\capitulum{Gal 5, 16-17}
{Spíritu ambuláte et concupiscéntiam carnis ne perfecéritis. / Caro enim
concupíscit advérsus Spíritum, Spíritus autem advérsus carnem ; † hæc
enim ínvicem adversántur, * ut non quæcúmque vultis illa faciátis.}
{Vivez sous la conduite de l’Esprit de Dieu ; alors vous n’obéirez pas aux
tendances égoïstes de la chair. Car les tendances de la chair s’opposent à
l’esprit, et les tendances de l’esprit s’opposent à la chair. En effet, il y a là
un affrontement qui vous empêche de faire ce que vous voudriez.}

\versiculus{Bonus es tu, Dómine, et benefáciens}{Doce me iustificatiónes tuas}{Toi, tu es bon, Seigneur, tu fais du bien}{Apprends-moi tes commandements}

\rubric{Oraison TODO}

\gscore{or_benedicamus_hm_festivus}

\bigtitle{Jeudi II, à None}{jeudi 10 juillet}{None}

\rubric{Ouverture, Hymne et Psalmodie de None, comme au mardi, p. \pageref{hy_rerum_deus_ferialis}.}

\capitulum{Gal 5, 22-23a.25}
{Fructus Spíritus est cáritas, gáudium, pax, longanímitas, * benígnitas,
bónitas, fides, mansuetúdo, continéntia. / Si vívimus Spíritu, Spíritu et
ambulémus.}
{Voici ce que produit l’Esprit : amour, joie, paix, patience, bonté,
bienveillance, foi, humilité et maîtrise de soi. Puisque l’Esprit nous fait
vivre, laissons-nous conduire par l’Esprit.}

\versiculus{Notam fac mihi, Dómine, viam in qua ámbulem}{Spíritus tuus bonus dedúcat me in terram rectam}{Montre-moi le chemin que je dois prendre}{Ton souffle est bienfaisant : qu’il me guide en un pays de plaines}

\rubric{Oraison TODO}

\gscore{or_benedicamus_hm_festivus}

\needspace{5cm}
\bigtitle{Saint Benoît, aux 1\textsuperscript{e} Vêpres}{jeudi 10 juillet}{Vêpres}

\smallscore{or_dia_festivus}

\gscore{hy_fratres_alacri_pectore}
\translation{\colored{F}rères, venez avec un cœur joyeux, partageons d'une même voix les joies de cette glorieuse fête.\\
\colored{E}n ce jour, Benoît, révélateur du chemin étroit, reçoit la récompense de ses travaux et se réjouit de l'honneur qui lui est dû.\\
\colored{I}l a brillé comme une nouvelle étoile, dissipant les nuages terrestres ; dès le seuil de l'âge, il a méprisé les plaisirs de la jeunesse.\\
\colored{P}uissant en miracles, inspiré par le souffle du Très-Haut, il a resplendi par des prodiges, annonçant les choses futures à son siècle.\\
\colored{P}armi ces œuvres, il a également brillé par ses enseignements lumineux, car il a tracé avec justesse la voie de la vie sacrée pour ceux qui le suivent.
\colored{G}loire à la Trinité, dont la grâce remplit ce saint de nombreuses fleurs de sainteté, embellissant ainsi les cieux.}

\needspace{3cm}
\smalltitle{Psaume 109}
\gscore{an_isti_sunt_viri_sancti}
\translation{\aa Ce sont eux les hommes saints
que le Seigneur a choisis dans une charité non feinte,
et il leur donna une gloire éternelle;
par leur doctrine, l’Eglise resplendit comme la lune par le soleil.}
\psalm{109}{7}

\smalltitle{Psaume 112}
\gscore{an_vos_estis}
\translation{\aa C'est vous qui êtes demeurés avec moi dans mes épreuves ;
et moi, je suis au milieu de vous comme celui qui sert.}
\psalm{112}{2}

\smalltitle{Psaume 115}
\gscore{an_isti_viventes_in_carne}
\translation{\aa Voici ceux qui dans leur vie terrestre ont planté l'Eglise par leur sang; leurs corps n'ont pas été enlevés de la terre, eux dont les mérites sont dans les cieux comme les âmes des saints.}
\psalm{115}{7}

\smalltitle{Psaume 125}
\gscore{an_iam_non_dicam_vos_servos}
\translation{\aa Je ne vous appelle plus serviteurs mais mes amis
car tout ce que j'ai appris de mon Père,
je vous l'ai fait connaître.}
\psalm{125}{1d}

\capitulum{Ep 4, 11-13}
{Christus dedit quosdam quidem apóstolos, quosdam autem
prophétas, † álios vero evangelístas, álios autem pastóres et doctóres *
ad instructiónem sanctórum in opus ministérii, in ædificatiónem córporis
Christi, † donec occurrámus omnes in unitátem fídei et agnitiónis Fílii
Dei, * in virum perféctum, in mensúram ætátis plenitúdinis Christi.}
{Les dons que le Christ a faits, ce sont les Apôtres, et aussi les prophètes,
les évangélisateurs, les pasteurs et ceux qui enseignent. De cette
manière, les fidèles sont organisés pour que les tâches du ministère
soient accomplies et que se construise le corps du Christ, jusqu’à ce
que nous parvenions tous ensemble à l’unité dans la foi et la pleine
connaissance du Fils de Dieu, à l’état de l’Homme parfait, à la stature
du Christ dans sa plénitude.}

\gscore{rb_annuntiate_inter_gentes}
\translation{\rr Racontez à tous les peuples la gloire du Seigneur.
\vv À toutes les nations ses merveilles.}

\newpage
\smalltitle{Magnificat}
\gscore{an_quicumque_voluerit_inter_vos}
\translation{\aa Celui qui veut devenir grand parmi vous sera votre serviteur ;
et celui qui veut être parmi vous le premier sera votre esclave.}
\incipit{magn}{8gsol}
\psalm{magn}{8sol}
\gscore{an_quicumque_voluerit_inter_vos}

\smalltitle{Intercession}
\smallscore{or_kyrie_festivus}
\smallscore{or_pater_festivus}

\label{0725V}
\oratio{
Omnípotens sempitérne Deus, qui Apostolórum tuórum primítias
beáti Iacóbi sánguine dedicásti, † da, quǽsumus, Ecclésiæ tuæ ipsíus
confessióne firmári, * et iúgiter patrocíniis confovéri. Per Dóminum.
}{
Dieu éternel et tout-puissant, tu as consacré l’offrande du bienheureux
Jacques, le premier de tes Apôtres à verser pour toi son sang, accorde à
ton Église de trouver dans son témoignage une force, et dans sa protection
un appui constant.
}

\blessing

\gscore{or_benedicamus_festis}

\end{document}