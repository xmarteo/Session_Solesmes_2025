% !TEX TS-program = lualatex
% !TEX encoding = UTF-8

\documentclass[Session2025.tex]{subfiles}

\ifcsname preamble@file\endcsname
  \setcounter{page}{\getpagerefnumber{M-02_lundi}}
\fi

\begin{document}
\addcontentsline{toc}{chapter}{Lundi 7 juillet, de la férie}
\bigtitle{Lundi de la 14\textsuperscript{e} semaine \emph{per annum, à la Messe}{lundi 7 juillet}{Messe}

\gscore{in_suscepimus}
\translation{\aa Nous avons reçu, ô Dieu, ta miséricorde au milieu de ton temple ; comme ton nom, ainsi ta louange retentit jusqu'aux confins de la terre ; ta droite est pleine de justice.\\
\vv Grand est le Seigneur, et très digne de louange, dans la cité de notre Dieu, sur sa montagne sainte.}

\smalltitle{Psalmodie de Tierce}
\gscore{an_adjuva_me}
\translation{\aa Aide-moi, et je serai sauvé, Seigneur.}
\smalltitle{Psaume 118, xiii}
\psalm{118-13}{8}
\smalltitle{Psaume 118, xiv}
\psalm{118-14}{8}
\smalltitle{Psaume 118, xv}
\psalm{118-15}{8}
\smalltitle{Psaume 118, xvi}
\psalm{118-16}{8}

\gscore{ky_k16}

\gscore{gr_angelis_suis}
\translation{\rr Il donne mission à ses anges de te garder sur tous tes chemins.\\
\vv Ils te porteront sur leurs mains, pour que ton pied de heurte pas les pierres.}
\gscore{al_magnus_dominus}
\translation{\vv Grand est le Seigneur, et très digne de louange, dans la cité de notre Dieu, sur sa montagne sainte.}
\gscore{of_populum_humilem}
\translation{\rr Tu sauveras le peuple humble, Seigneur, et tu humilieras les yeux des superbes ; car qui est Dieu sinon toi, Seigneur ?}
\gscore{ky_s16}
\gscore{ky_a16}
\gscore{co_gustate}
\translation{\aa Goûtez et voyez comme est doux le Seigneur: heureux l'homme qui espère en lui.\\
\vv Je bénirai le Seigneur en tous temps, sa louange sans cesse à ma bouche.\\
\vv Je me glorifierai dans le Seigneur, que les pauvres m'entendent et se réjouissent.}

\bigtitle{Lundi II, à Sexte}{lundi 7 juillet}{Sexte}

\smallscore{or_dia_ferialis}

\rubric{Hymne \normaltext{Rector potens}, ton férial, p. \pageref{hy_rector_potens_ferialis}.}

\gscore{an_adspice_in_me}
\translation{\aa Regarde-moi et aie pitié de moi, Seigneur.}
\smalltitle{Psaume 118, xvii}
\psalm{118-17}{1}
\smalltitle{Psaume 118, xvii}
\psalm{118-18}{1}
\smalltitle{Psaume 118, xix}
\psalm{118-19}{1}
\gscore{an_ardens_est_cor}

\capitulum{Jr 32, 40}
{Fériam eis pactum sempitérnum et non désinam eis benefácere * et
timórem meum dabo in corde eórum, ut non recédant a me.}
{Je conclus avec eux une Alliance éternelle : je ne cesse de les poursuivre
de mes bienfaits et je fais qu’ils me respectent profondément, sans plus
jamais s’écarter de moi.}

\versiculus{In Deo salutáre meum et glória mea}{Et refúgium meum in illo}{Mon salut et ma gloire se trouvent près de Dieu}{Chez Dieu, mon refuge}

\rubric{Oraison TODO}

\gscore{or_benedicamus_hm_ferialis}

\bigtitle{Lundi II, à None}{lundi 7 juillet}{None}

\smallscore{or_dia_ferialis}

\rubric{Hymne \normaltext{Rerum Deus}, ton férial, p. \pageref{hy_rector_potens_ferialis}.}

\gscore{an_fiat_manus_tua}
\translation{\aa Que ta main soit sur moi, Seigneur, pour que tu me sauves, car j'ai désiré tes commandements.}
\smalltitle{Psaume 118, xx}
\psalm{118-20}{4}
\needspace{3cm}
\smalltitle{Psaume 118, xxi}
\psalm{118-21}{4}
\smalltitle{Psaume 118, xxii}
\psalm{118-22}{4}

\capitulum{Ez 34, 31}
{Vos grex meus, grex páscuæ meæ vos, et ego Dóminus Deus vester, *
dicit Dóminus Deus.}
{Vous êtes mon troupeau, le troupeau de mon pâturage, vous les hommes.
Moi, je suis votre Dieu – oracle du Seigneur Dieu.}

\versiculus{Dóminus regit me et nihil mihi déerit}{In loco páscuæ ibi me collocávit}{Le Seigneur est mon berger : je ne manque de rien}{Sur des prés d’herbe fraîche, il me fait reposer}

\rubric{Oraison TODO}

\gscore{or_benedicamus_hm_ferialis}

\bigtitle{Lundi II, aux Vêpres}{lundi 7 juillet}{Vêpres}

\smallscore{or_dia_ferialis}

\gscore{hy_luminis_fons}
\translation{\colored{É}TODO\\
\colored{Q}uand se révèle à toi sa puissance redoutable qui chasse les forces démoniaques, pleine de reconnaissance pour celui qui t’a guérie,
tu te réjouis d’être maintenant enchaînée par le lien plus fort de la foi.\\
\colored{D}ès lors, ton amour te presse de te tenir aux pieds du Maître ; tu l’accompagnes et l’entoures avec ferveur de soins empressés.\\
\colored{A}vec celles qui pleurent le Seigneur, debout au pied de la croix, tu brûles d’une ardente tendresse ; 
tu laves avec amour et tu embaumes ses membres avant de les livrer au tombeau.\\
\colored{L}’amour du Christ nous a engendrés : fais que nous soyons unis à ton triomphe pour l’éternité,
et que nous chantions avec toi à profusion les louanges du Bien-Aimé.}

\smalltitle{Psaume 109}
\gscore{an_TODO}
\translation{\aa Le sabbat terminé, Marie Madeleine, Marie, mère de Jacques, et Salomé
achetèrent des parfums
pour aller embaumer le corps de Jésus, alléluia.}
\psalm{109}{4}

\smalltitle{Psaume 112}
\gscore{an_inclinavit_se_maria}
\translation{\aa Marie se pencha vers l’intérieur du tombeau,
et, elle aperçut deux anges,
assis, vêtus de blanc, alléluia.}
\psalm{112}{1}

\smalltitle{Psaume 121}
\gscore{an_ardens_est_cor}
\translation{\aa Le coeur brûlant, je désire voir mon Seigneur ;
je le cherche, et je ne trouve pas
où ils l’ont déposé, alléluia.}
\psalm{121}{8}

\smalltitle{Psaume 126}
\gscore{an_dicit_iesus_maria}
\translation{\aa Jésus dit alors : « Marie ! » Elle se tourne vers lui et lui dit :
« Rabbouni ! » ce qui veut dire : « Maître ».}
\psalm{126}{4}

\capitulum{1 Th 2, 13}
{Grátias ágimus Deo sine intermissióne, † quóniam cum accepissétis
a nobis verbum audítus Dei, * accepístis non ut verbum hóminum sed,
sicut est vere, verbum Dei, quod et operátur in vobis qui créditis.}
{Voici pourquoi nous ne cessons de rendre grâce à Dieu. Quand vous avez
reçu de notre bouche la parole de Dieu, vous l’avez accueillie pour ce
qu’elle est réellement : non pas une parole d’hommes, mais la parole de
Dieu qui est à l’oeuvre en vous, les croyants.}

\gscore{rb_tibi_dixit} % TODO changer
\translation{\rr Mon coeur t’a dit : J’ai recherché ta face.\\
\vv C’est ta face, Seigneur, que je cherche.}

\needspace{5cm}
\smalltitle{Magnificat}
\gscore{an_venit_maria_nuntians}
\translation{\aa Marie Madeleine s’en va annoncer aux disciples :
« J’ai vu le Seigneur, alléluia. »}
\incipit{magn}{7asol}
\psalm{magn}{7sol}

\smalltitle{Intercession}
\smallscore{or_kyrie_ferialis}
\smallscore{or_pater_ferialis}

\label{0722V}
\oratio{
TODO
}{
Seigneur Dieu, c’est à Marie Madeleine que ton Fils unique a confié la
première annonce de la joie pascale ; accorde-nous, à sa prière et à son
exemple, de proclamer que le Christ est vivant et de le contempler dans
la gloire de ton Royaume.
}
\newpage
\blessing

\gscore{or_benedicamus_festis}

\end{document}