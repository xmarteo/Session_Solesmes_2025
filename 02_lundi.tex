% !TEX TS-program = lualatex
% !TEX encoding = UTF-8

\documentclass[Session2025.tex]{subfiles}

\ifcsname preamble@file\endcsname
  \setcounter{page}{\getpagerefnumber{M-02_lundi}}
\fi

\begin{document}
\addcontentsline{toc}{chapter}{Lundi 7 juillet, de la férie}
\bigtitle{Lundi de la 14\textsuperscript{e} semaine \emph{per annum, à la Messe}{lundi 7 juillet}{Messe}

\gscore{in_suscepimus}
\translation{\aa Nous avons reçu, ô Dieu, ta miséricorde au milieu de ton temple ; comme ton nom, ainsi ta louange retentit jusqu'aux confins de la terre ; ta droite est pleine de justice.\\
\vv Grand est le Seigneur, et très digne de louange, dans la cité de notre Dieu, sur sa montagne sainte.}

\smalltitle{Psalmodie de Tierce}
\gscore{an_adjuva_me}
\translation{\aa Aide-moi, et je serai sauvé, Seigneur.}
\smalltitle{Psaume 118, xiii}
\psalm{118-13}{8}
\smalltitle{Psaume 118, xiv}
\psalm{118-14}{8}
\smalltitle{Psaume 118, xv}
\psalm{118-15}{8}
\smalltitle{Psaume 118, xvi}
\psalm{118-16}{8}

\gscore{ky_k16}

\gscore{gr_angelis_suis}
\translation{\rr Il donne mission à ses anges de te garder sur tous tes chemins.\\
\vv Ils te porteront sur leurs mains, pour que ton pied de heurte pas les pierres.}
\gscore{al_magnus_dominus}
\translation{\vv Grand est le Seigneur, et très digne de louange, dans la cité de notre Dieu, sur sa montagne sainte.}
\gscore{of_populum_humilem}
\translation{\rr Tu sauveras le peuple humble, Seigneur, et tu humilieras les yeux des superbes ; car qui est Dieu sinon toi, Seigneur ?}
\gscore{ky_s16}
\gscore{ky_a16}
\gscore{co_gustate}
\translation{\aa Goûtez et voyez comme est doux le Seigneur: heureux l'homme qui espère en lui.\\
\vv Je bénirai le Seigneur en tous temps, sa louange sans cesse à ma bouche.\\
\vv Je me glorifierai dans le Seigneur, que les pauvres m'entendent et se réjouissent.}

\bigtitle{Lundi II, à Sexte}{lundi 7 juillet}{Sexte}

\smallscore{or_dia_ferialis}

\rubric{Hymne \normaltext{Rector potens}, ton férial, p. \pageref{hy_rector_potens_ferialis}.}

\gscore{an_adspice_in_me}
\translation{\aa Regarde-moi et aie pitié de moi, Seigneur.}
\smalltitle{Psaume 118, xvii}
\psalm{118-17}{1}
\smalltitle{Psaume 118, xvii}
\psalm{118-18}{1}
\smalltitle{Psaume 118, xix}
\psalm{118-19}{1}
\gscore{an_ardens_est_cor}

\capitulum{Jr 32, 40}
{Fériam eis pactum sempitérnum et non désinam eis benefácere * et
timórem meum dabo in corde eórum, ut non recédant a me.}
{Je conclus avec eux une Alliance éternelle : je ne cesse de les poursuivre
de mes bienfaits et je fais qu’ils me respectent profondément, sans plus
jamais s’écarter de moi.}

\versiculus{In Deo salutáre meum et glória mea}{Et refúgium meum in illo}{Mon salut et ma gloire se trouvent près de Dieu}{Chez Dieu, mon refuge}

Deus, qui messis ac víneæ dóminus es et custos,
quique offícia tríbuis et iusta stipéndia meritórum, fac
nos diéi pondus ita portáre, ut nihil umquam de tuis
plácitis conquerámur. Per Christum.

Maître de la vigne et de la
moisson, toi qui répartis
les tâches et donnes le vrai salaire,
aide-nous à porter le poids
du jour sans murmurer contre ta
volonté. Par le Christ.

\gscore{or_benedicamus_hm_ferialis}

\bigtitle{Lundi II, à None}{lundi 7 juillet}{None}

\smallscore{or_dia_ferialis}

\rubric{Hymne \normaltext{Rerum Deus}, ton férial, p. \pageref{hy_rector_potens_ferialis}.}

\gscore{an_fiat_manus_tua}
\translation{\aa Que ta main soit sur moi, Seigneur, pour que tu me sauves, car j'ai désiré tes commandements.}
\smalltitle{Psaume 118, xx}
\psalm{118-20}{4}
\needspace{3cm}
\smalltitle{Psaume 118, xxi}
\psalm{118-21}{4}
\smalltitle{Psaume 118, xxii}
\psalm{118-22}{4}

\capitulum{Ez 34, 31}
{Vos grex meus, grex páscuæ meæ vos, et ego Dóminus Deus vester, *
dicit Dóminus Deus.}
{Vous êtes mon troupeau, le troupeau de mon pâturage, vous les hommes.
Moi, je suis votre Dieu – oracle du Seigneur Dieu.}

\versiculus{Dóminus regit me et nihil mihi déerit}{In loco páscuæ ibi me collocávit}{Le Seigneur est mon berger : je ne manque de rien}{Sur des prés d’herbe fraîche, il me fait reposer}

Deus, qui nos ádvocas illa hora, qua ad templum
ascendébant Apóstoli, præsta, ut orátio, quam in nómine
Iesu sincéra tibi mente persólvimus, ómnibus
nomen illud invocántibus salútem eius váleat impetrare.
Per Christum.

Tu nous invites, Seigneur,
à nous réunir près de toi
comme les Apôtres qui montaient
au Temple à la neuvième
heure. Que notre prière faite au
nom de Jésus appelle ton salut
sur tous ceux qui invoquent son
nom. Lui qui.

\gscore{or_benedicamus_hm_ferialis}

\bigtitle{Lundi II, aux Vêpres}{lundi 7 juillet}{Vêpres}

\smallscore{or_dia_ferialis}

\smalltitle{Psaume 113}
\gscore{an_nos_qui_vivimus}
\translation{\aa Nous qui vivons, bénissons le Seigneur.}
\psalm{113}{p}

\smalltitle{Psaume 114}
\gscore{an_inclinavit_dominus}
\translation{\aa Le Seigneur a incliné Son oreille vers moi.}
\psalm{114}{1}

\smalltitle{Psaume 115}
\gscore{an_credidi}
\translation{\aa J'ai cru, c'est pourquoi je parle.}
\psalm{115}{8}
\rubric{On ne répète pas l'antienne et on n'ajoute pas \normaltext{Gloria Patri}.}
\smalltitle{Psaume 116}
\psalm{116}{8}

\smalltitle{Psaume 128}
\gscore{an_saepe_expugnaverunt}
\translation{\aa Ils m'ont souvent attaqué depuis ma jeunesse.}
\psalm{128}{4}

\capitulum{1 Th 2, 13}
{Grátias ágimus Deo sine intermissióne, † quóniam cum accepissétis
a nobis verbum audítus Dei, * accepístis non ut verbum hóminum sed,
sicut est vere, verbum Dei, quod et operátur in vobis qui créditis.}
{Voici pourquoi nous ne cessons de rendre grâce à Dieu. Quand vous avez
reçu de notre bouche la parole de Dieu, vous l’avez accueillie pour ce
qu’elle est réellement : non pas une parole d’hommes, mais la parole de
Dieu qui est à l’oeuvre en vous, les croyants.}

\gscore{rb_dirigatur}
\translation{\rr Que ma prière, Seigneur, * soit dirigée vers toi.\\
\vv Comme l'encens en ta présence.}

\gscore{hy_luminis_fons}
\translation{\colored{L}umière, origine et source de la lumière,
ô Dieu très bon, sois favorable à nos prières :
repousse au loin l’obscurité de nos péchés,
embellis-nous de ta lumière.\\
\colored{V}oici qu’est achevé le labeur de ce jour,
de par ta volonté nous sommes sains et saufs ;
vois, de tout notre coeur nous te disons merci,
dès maintenant et pour toujours.\\
\colored{L}e coucher du soleil ramène la ténèbre ;
mais qu’il brille pour nous, l’étincelant soleil
qui embrase là-haut, de sa lumière d’or,
la multitude des saints anges.\\
\colored{T}out ce que la journée a pu cacher de fautes, 
que le Christ indulgent et très bon le détruise : 
alors un pur éclat fera luire nos coeurs
au plus profond de cette nuit.\\
\colored{L}ouange à toi, le Père, et gloire à toi, le Fils,
semblable seigneurerie à leur Souffle très saint,
ô vous qui gouvernez avec autorité
le monde à travers tous les siècles.}

\versiculus{Vespertína orátio ascéndat ad te, Dómine}{Et descéndat super nos misericórdia tua}{Que la prière du soir monte vers toi, Seigneur}{Et que descende sur nous ta miséricorde}

\smalltitle{Magnificat}
\gscore{an_magnificet_te_semper}
\translation{\aa Que mon âme toujours Te magnifie, mon Dieu.}
\incipit{magn}{4star}
\psalm{magn}{4star}

\smalltitle{Intercession}
\smallscore{or_kyrie_ferialis}
\smallscore{or_pater_ferialis}

\label{0722V}
\oratio{
Deus omnípotens, qui hódie servos tuos inútiles in
labóribus roborásti, hoc laudis súscipe, quod tibi offérimus,
sacrifícium vespertínum, de suscéptis a te
munéribus grátias referéntes. Per Dóminum.
}{
À l’heure du sacrifice du soir,
nous nous présentons devant
toi, Seigneur, comme des
serviteurs inutiles, mais recueille
sur nos lèvres l’action de grâce
pour tous les biens reçus de toi.
Par Jésus Christ.
}
\blessing

\gscore{or_benedicamus_festis}

\end{document}