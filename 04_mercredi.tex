% !TEX TS-program = lualatex
% !TEX encoding = UTF-8

\documentclass[Session2025.tex]{subfiles}

\ifcsname preamble@file\endcsname
  \setcounter{page}{\getpagerefnumber{M-04_mercredi}}
\fi

\begin{document}

\addcontentsline{toc}{chapter}{Mercredi 9 juillet, Vierge Marie, Mère de Providence}
\bigtitle{Vierge Marie, Mère de Providence,\\à la Messe}{mercredi 9 juillet}{Messe}

\gscore{in_vultum_tuum}
\translation{\aa Tous les riches du peuple imploreront votre visage. On amènera au Roi des vierges à sa suite: ses proches vous seront présentées dans l’allégresse.\\
\vv D'heureuses paroles jaillissent de mon coeur quand je dis mes poèmes pour le roi.}

\vfill
\rubric{Psalmodie de Tierce, comme au mardi, p. \pageref{an_clamavi}.}
\vfill

\newpage

\smalltitle{Kyrie ad lib. VIII}
\gscore{ky_k8adlib}
\vspace{50pt}
\gscore{gr_adjutor_in_opportunitatibus}
\translation{\rr Secours au temps voulu, dans la détresse :
qu’ils mettent en vous leur espérance,
ceux qui vous connaissent :
car vous n’abandonnez pas ceux qui vous cherchent, Seigneur.\\
\vv Car le pauvre n'est pas oublié pour toujours: jamais ne périt l'espoir des malheureux. Lève-toi, Seigneur: qu'un mortel ne soit pas le plus fort.}

\newpage

\gscore{al_domine_refugium}
\translation{\vv Le Seigneur fut pour nous un refuge, de génération en génération.}

\gscore{of_ave_maria}

\smalltitle{Sanctus X}
\gscore{ky_s10}
\smalltitle{Agnus X}
\gscore{ky_a10}
\gscore{co_laetabimur}
\translation{\aa Nous nous réjouirons de ton salut, nous serons glorifiés dans le nom du Seigneur notre Dieu.\\
\vv Que le Seigneur te réponde au jour de détresse, que le nom du Dieu de Jacob te défende.\\
\vv Du sanctuaire, qu'il t'envoie le secours, qu'il te soutienne des hauteurs de Sion.}

\bigtitle{Vierge Marie, Mère de Providence, à Sexte}{mercredi 9 juillet}{Sexte}

\smallscore{or_dia_ferialis}

\gscore{hy_rector_potens_memoriis}
\translation{\colored{M}aître puissant, Dieu Vérité, tu règles la marche du temps, tu formes l’aube en sa clarté, au midi tu donnes ses flammes.\\
\colored{E}teins le feu des dissensions, calme la fièvre du péché, apporte à nos corps la santé, à nos cœurs, la paix véritable.\\
\colored{E}xauce-nous, Père très bon, et toi, le Fils égal au Père, avec l’Esprit Consolateur, régnant pour les siècles des siècles.}

\rubric{Psalmodie de Sexte, comme au mardi, p. \pageref{an_qui_habitas}.}

\pagebreak

\capitulum{Col 3, 17}
{Omne quodcúmque fácitis in verbo aut in ópere, † ómnia in nómine
Dómini Iesu * grátias agéntes Deo Patri per ipsum.}
{Tout ce que vous dites, tout ce que vous faites, que ce soit toujours au
nom du Seigneur Jésus-Christ, en offrant par lui votre action de grâce à
Dieu le Père.}

\versiculus{Tibi, Dómine, sacrificábo hóstiam laudis}{Et nomen Dómini invocábo}{Je t’offrirai, Seigneur, le sacrifice d’action de grâce}{J’invoquerai le nom du Seigneur}

\rubric{Oraison des Vêpres, p. \pageref{0709_oratio}.}

\gscore{or_benedicamus_hm_ferialis}

\bigtitle{Vierge Marie, Mère de Providence, à None}{mercredi 9 juillet}{None}

\smallscore{or_dia_ferialis}

\gscore{hy_rerum_deus_memoriis}
\translation{\colored{D}ieu fort, soutien de l’univers, qui es en toi sans changement, tu fixes dans leur succession les temps que le soleil mesure.\\
\colored{D}onne-nous un soir lumineux où la vie ne décline pas, où, pour fruit de la Sainte Mort, resplendira toujours la gloire.\\
\colored{E}xauce-nous, Père très bon, et toi, le Fils égal au Père, avec l’Esprit Consolateur, régnant pour les siècles des siècles.}

\rubric{Psalmodie de None, comme au mardi, p. \pageref{an_beati_omnes}.}

\capitulum{Col 3, 23-24}
{Quodcúmque fácitis, ex ánimo operámini sicut Dómino et non
homínibus, * sciéntes quod a Dómino accipiétis retributiónem hereditátis. / Dómino Christo servíte.}
{Quel que soit votre travail, faites-le de bon coeur, pour le Seigneur et non
pour plaire à des hommes : vous savez bien qu’en retour le Seigneur fera
de vous ses héritiers. Le maître c’est le Christ : vous êtes à son service.}

\versiculus{Dóminus pars hereditátis meæ et cálicis mei}{Tu es qui détines sortem meam}{Seigneur, mon partage et ma coupe}{De toi dépend mon sort}

\rubric{Oraison des Vêpres, p. \pageref{0709_oratio}.}

\gscore{or_benedicamus_hm_ferialis}

\pagebreak

\bigtitle{Vierge Marie, Mère de Providence,\\aux Vêpres}{mercredi 9 juillet}{Vêpres}

\smallscore{or_dia_ferialis}

\smalltitle{Psaume 134}
\gscore{an_omnia_quaecumque}
\translation{\aa Tout ce qu'il veut, le Seigneur le fait.}
\psalm{134}{3}

\smalltitle{Psaume 135}
\gscore{an_quoniam_in_aeternum}
\translation{\aa Car éternel est son amour.}
\psalm{135}{3}

\smalltitle{Psaume 136}
\gscore{an_hymnum_cantate}
\translation{\aa Chantez-nous un chant de Sion.}
\psalm{136}{8}

\smalltitle{Psaume 137}
\gscore{an_in_conspectu_angelorum}
\translation{\aa Mon dieu, je te chanterai des psaumes en présence des anges.}
\psalm{137}{5}

\capitulum{1 Jn 2, 3-6}
{In hoc cognóscimus quóniam nóvimus Christum : * si mandáta eius
servémus. / Qui dicit : « Novi eum », et mandáta eius non servat, mendax
est, et in isto véritas non est ; † qui autem servat verbum eius, * vere in
hoc cáritas Dei consummáta est. / In hoc cognóscimus quóniam in ipso
sumus. / Qui dicit se in ipso manére, debet, sicut ille ambulávit, et ipse
ambuláre.}
{Voici comment nous pouvons savoir que nous connaissons Jésus-Christ :
c’est en gardant ses commandements. Celui qui dit : « Je le connais »,
et qui ne garde pas ses commandements, est un menteur : la vérité n’est
pas en lui. Mais en celui qui garde fidèlement sa parole, l’amour de Dieu
atteint vraiment la perfection : voilà comment nous reconnaissons que
nous sommes en lui. Celui qui déclare demeurer en lui doit marcher lui-même
dans la voie où lui, Jésus, a marché.}

\gscore{rb_custodi_nos}
\translation{\rr Garde-nous, Seigneur, comme la prunelle de l'œil.\\
\vv Protège-nous à l'ombre de tes ailes.}

\gscore{hy_o_quam_glorifica}
\translation{\colored{D}e quelle glorieuse lumière vous brillez, royale enfant de la lignée de David, Vierge Marie, vous qui siègez très haut au dessus de tous les habitants des cieux.\\
\colored{M}ère qui gardez votre honneur de vierge, vous avez préparé chastement, pour le Seigneur des cieux, dans votre sein consacré, la demeure de votre cœur, d'où naquit le Christ, Dieu incarné.\\
\colored{C}elui qu'adore avec vénération tout l'univers, devant qui à bon droit tout genou fléchit, nous implorons de Lui, par votre secours, les joies de la lumière qui chasse les ténèbres. \\
\colored{F}ais-nous cette largesse, Père de toute lumière, dans l'Esprit Saint, par Ton propre Fils qui, vivant avec Toi dans l'éclat de l'éther, règne et gouverne tous les siècles.}

\versiculus{Diffúsa est grátia in lábiis tuis}{Proptérea benedíxit te Deus in ætérnum}{La grâce est répandue sur vos lèvres}{Que Dieu vous bénisse pour toujours}

\smalltitle{Magnificat}
\gscore{an_exsultavit_quia_fecit}
\translation{\aa Mon esprit exulte en Dieu mon sauveur, car le puissant fit pour moi des merveilles.}
\incipit{magn}{6f}
\psalm{magn}{6}

\needspace{3cm}
\smalltitle{Intercession}
\smallscore{or_kyrie_ferialis}
\smallscore{or_pater_ferialis}

\label{0709_oratio}
\oratio{Omnípotens, sempitérne Deus, Ecclésiam tuam, déxtera poténtiæ tuæ, a cunctis prótege perículis,~\pscross et beáta María Matre Providéntiæ intercedénte,~\psstar fac eam præsénti gaudére prosperitáte et futúra. Per Dóminum.}{Dieu éternel et tout puissant, protège ton Église de tous les dangers par la force de ton bras, et par l'intercession de la bienheureuse Marie, Mère de Providence, fais-la se réjouir de sa prospérité présente et future. Par Jésus-Christ.}

\blessing

\gscore{or_benedicamus_festis}

\end{document}
